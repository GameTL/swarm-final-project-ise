\chapter{Project Contribution per Student}

\paragraph*{}
In the first phase, the project tasks are divided among the group members to ensure a focused effort on simulating the fundamental components. Two group members will concentrate on developing the communication aspect, ensuring that the robots can coordinate and communicate effectively within the swarm. Simultaneously, one member will focus on object detection using computer vision techniques, while another two members tackle the basic implementation of SLAM (Simultaneous Localization and Mapping). This phase is scheduled to span from October to early November.

\paragraph*{}
Moving into the second phase, which runs from early November to early December, the project transitions to integrating hardware and enhancing SLAM functionalities. Initially, all group members will collaborate on incorporating simple movement into the hardware, ensuring a robust foundation for the next tasks. One group member will then test the SLAM implementation in two different environments to verify its effectiveness. Concurrently, two members will work on improving SLAM by developing a more advanced version, such as C-SLAM. After completing these tasks, the dedicated SLAM team, composed of three members, will integrate the individual modules from the first phase into a cohesive simulation, aiming to achieve a Minimum Viable Prototype by the semester's end.

\paragraph*{}
In the third phase, from January to early March of the second semester, the focus shifts to more advanced capabilities, including gripping mechanisms and coordinated movement. Two group members will work on the coordinated gripping task, preparing the static gripper for future sliding movements. Meanwhile, another three members will develop strategies for robot formation, ensuring that the robots can move together seamlessly after gripping objects.

\paragraph*{}
Finally, the fourth phase, starting in March and extending to early May, involves putting all the functional simulation components together with the actual robotic hardware. During this period, all group members will collaborate intensively to integrate these components, ensuring they work together smoothly in the real world.
