\chapter{Research Background  \markboth {Research Background}{}}
%\markboth {General introduction}{}}
%\addcontentsline{toc}{chapter}{General introduction}

\paragraph*{}
In recent years, swarm robotics has gained significant popularity in mobile robotics. Researchers have explored various aspects to create robot swarms, focusing on communication between the robots, the coordination of each robot, navigation, and the realization of animal-inspired behaviors in robotic systems \cite{cheraghi2021past}. Additionally, certain properties exhibited by natural swarms have been identified as crucial for swarm robotics. G. Beni proposes the following properties: Flexibility, defined as "the capability to adapt to new, different, or changing requirements of the environment" \cite{bayindir2007review}; Scalability, where "a swarm system is said to be scalable if it can work with different numbers of its members" \cite{nedjah2019review}; and Robustness, "the ability of a swarm robotic system to continue operating, although at a lower performance, despite disturbances in the environment or failures in individuals" \cite{sahin_swarm_2005}. However, additional properties are essential for swarm robotics systems that researchers address when designing their robots. These properties include Autonomy, the ability of individual robots within a swarm to operate independently while coordinating with others to achieve a common goal; Self-organization, "a process whereby pattern at the global level of a system emerges solely from interactions among the lower levels of the system... using only local information, without any central authority determining their course of action" \cite{cheraghi2021past}; Self-assembly, where "the robots not only stay close to each other, but they also are able to connect themselves, forming a single organism" \cite{nedjah2019review}; Decentralization, where "each individual makes its own decisions based on local information" \cite{koifman2024distributed}; and Stigmergy, described as "a form of indirect communication between natural or artificial agents where the work performed by an agent leaves a trace in the environment that stimulates the performance of subsequent work by the same or other agents" \cite{cheraghi2021past}.
 

\paragraph*{}
Given the current state of the art, it is still impossible to create a perfect swarm robot system that perfectly manifests such properties, let alone the challenges of dealing with the sub-system inside the robot system itself, for example, the implementation of Simultaneous Localization and Mapping (SLAM) on robot navigation. Moreover, the metrics for evaluating swarm SLAM methods are impractical in real-life situations; one might make arbitrary decisions on the quantification of aspects \cite{kegeleirs2021swarm}. 

\paragraph*{}
This project aims to address the communication, coordination, and navigation challenges within decentralised swarm robot systems. By enabling the robots to share environmental data and make collective decisions in real time, this framework can improve flexibility, scalability, and adaptability in diverse industrial applications. 