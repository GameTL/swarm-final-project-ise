\chapter{Summary and Benefits to Real Industry}

\paragraph*{}
It’s becoming increasingly inevitable that robots will take over a significant portion of the workforce in the near future, especially for repetitive and monotonous tasks. As industries like warehousing, manufacturing, and even domestic environments integrate more robots, the ability for multiple robots to work together in the same space becomes critical. This is where swarm robotics comes into play. In such systems, robots must be able to communicate effectively to avoid collisions, coordinate task allocation, and share progress, ensuring smooth and efficient workflows. This project is a step toward achieving that goal by advancing decentralised, collective robotic behaviour.

\paragraph*{}
A key feature of this project is the decentralised communication system, where robots can interact with one another without needing a host computer or central control. This makes the system incredibly flexible and easy to deploy in various settings—whether in the field or at fixed locations like warehouses or homes. The decentralised nature allows for a plug-and-play style of operation, eliminating the need for complex initial setups. The mesh communication system also provides resilience, as the robots can continue working even if one or more units experience failure, ensuring task completion. This makes the system well-suited not just for static environments but also for dynamic, outdoor applications where flexibility is key.

\paragraph*{}
The benefits extend across various industries, from cleaning and tidying in commercial or domestic spaces (homes, offices, retail) to more complex environments like schools or factories where multiple robots operate together. The technology’s adaptability and resilience make it applicable in any scenario where more than one or two robots are required to collaborate, making it a valuable solution for the future of automation.
