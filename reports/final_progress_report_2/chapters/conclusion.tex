\chapter{Conclusion}

\paragraph*{}
The progress for implementing effective communication in the swarm has been completed except for the final integration with the actual swarm. In addition, the path planning algorithm that has been implemented allows for the swarm to be able to move and coordinate with each other without colliding within the swarm, as well as avoid any detected obstacles.

\paragraph*{}
Our hardware improvements have significantly enhanced robot performance. Metal shafts replaced 3D printed ones, reducing wobble, while thicker rubber omni-wheels decreased slippage. We've successfully mounted the LiDAR and camera with minimal parallax error. Future work includes empirical testing of these improvements and exploring an acrylic base alternative to the current PLA design. Overall, we've established a stable platform ready for the next phase of integration.

\paragraph*{}
Additionally, the object detection component remains on schedule, as the key aspects, including the detection model based on HSV and the extraction of distance measurements from the LiDAR, have been successfully completed. However, calibration and object occlusion handling will be implemented and tested once the necessary hardware is fully prepared for experimentation.

\paragraph*{}
For SLAM we will continue making our own graph SLAM but will work on SLAM Toolbox temporarily. For odometry, we will continue testing and publish results soon.

\paragraph*{}
Meanwhile, the development and testing of our holonomic X-Drive robot have progressed significantly. The robot’s base and structural components were 3D printed using PLA, but to improve durability and stability, key parts such as the baseplate, shaft, wheels, and flange coupler will be upgraded to acrylic and stainless steel. The robot utilizes Dynamixel AX-12W motors for enhanced back-drivability, controlled through a custom ROS 2 package that applies a Jacobian matrix transformation for motion control. Transitioning from the older Maxon BLDC motors has streamlined development, allowing us to focus on software for coordinated multi-robot operation. The successful integration of the teleop keyboard package in ROS 2 Humble has demonstrated holonomic movement, paving the way for further improvements in geometry and performance optimization.

\paragraph*{}
For the next phase of our project, we will complete the tasks that have yet to be completed during the previous iteration and move towards a complete swarm. This includes completing the hardware requirements, considering movement after gripping, as well as testing and evaluation. Additionally, we will proceed with other tasks as scheduled in the Gantt Chart to ensure all project milestones are met.
