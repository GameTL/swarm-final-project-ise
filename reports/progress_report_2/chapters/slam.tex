\chapter{Choice of SLAM}

\paragraph*{}
In our last progress report we mentioned that we had a very basic implementation of SLAM working. This however is only true in some aspects. While we do have a basic implementation of localisation, we do not currently have the ability to do simultaneous mapping of an unknown environment which was a mistake on our part. In the following weeks, we first attempted to implement SLAM through the use of ROS2 so we could easily access SLAM libraries like gmapping and cartographer . We specifically used ROS2 with the turtlebot3burger all integrated together in webots. However; we faced problems when it came to the communication between ROS2 and the simulation due to our limited experience with the aforementioned software. We then decided to pursue a SLAM option that does not depend on ROS such as EKF SLAM and Graph SLAM. EKF SLAM initially seemed familiar to us as we recently learnt the Kalman filter. However; we decided to go with graph SLAM for numerous reasons. Graph SLAM is more accurate than EKF SLAM because it performs global optimization by maintaining a graph of all robot poses and measurements, allowing for efficient correction of accumulated errors, especially during loop closure when the robot revisits known locations. This could be especially helpful in our project since our robots after placing the object in its desired location will be coming back to the main map. Unlike EKF SLAM, which uses linearization and only updates the current pose, Graph SLAM adjusts the entire trajectory and map, minimising drift and non-linearities. 