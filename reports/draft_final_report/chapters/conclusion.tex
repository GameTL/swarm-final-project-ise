\chapter{Conclusion}

\paragraph*{}
Swarm communication has been successfully implemented using a decentralized peer-to-peer architecture facilitated by socket communication. This approach was selected for its low latency and reliability, both of which are essential for real-time data exchange among robots. A three-way handshake mechanism was incorporated to ensure message delivery integrity, along with a consensus algorithm to resolve simultaneous object detection and taskmaster assignment conflicts. With coordinate streaming, dynamic path planning, and task execution now fully integrated, the communication system provides a solid foundation for coordinated swarm behavior, enabling seamless transitions between detection, planning, and movement phases.

\paragraph*{}
Additionally, the object detection system utilizing multi-sensor fusion of the camera and S3 RPLiDAR accurately measures relative distance, relative angle, and estimated width, aided by the variation factor function. This function yields an error margin of 1 centimeter for cylinders with diameters of 16 and 20 centimeters. The model was also tested with objects of different sizes, specifically a cylinder with a diameter of 12 centimeters, resulting in a slightly higher error margin of 2 centimeters. Overall, the object detection component was completed within the expected timeline.

\paragraph*{}
During the implementation stage, we noticed constraints on the Dynamixel AX-12W motor encoder. This experience showed the importance of thoroughly reviewing hardware documentation prior to system integration. Misinterpretation of sensor capabilities can significantly reduce system performance and reliability. However, we created an overhead camera system with ArUco markers, which the system effectively extracted. In our tests, this method exceeded LiDAR in terms of consistency and accuracy. This case displays the value of flexible problem solving. Using ArUco-based vision as a temporary but effective solution, we were able to deliver accurate ground-truth positioning during the testing and evaluation phases.
\paragraph*{}
On the hardware side we successfully transitioned from a CAD model into a functional physical prototype. Through iterative development, components such as the base frame, shaft supports, and custom sensor mounts were prototyped using PLA+ and precision-calibrated with tools like vernier calipers. The use of a Bambu Lab P1S printer allowed high-accuracy press-fit assemblies, particularly for bearing integration and shaft alignment.

The initial drive system was built around Dynamixel AX-12W motors, which provided modular control but suffered from limitations in encoder accuracy—posing challenges for precise odometry. To address this, work has begun on a redesigned base platform using DC motors with optical encoders for improved feedback, driven by an ESP32 microcontroller. While this new platform is not yet fully integrated, it demonstrates a forward-looking shift toward higher control fidelity and better hardware-software integration.
\paragraph*{}
In summary, this project successfully demonstrates the core principles of decentralized swarm robotics, providing a solid foundation for future enhancements in control, coordination, and scalability.
