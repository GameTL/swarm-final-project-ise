\chapter{Project Concept Development}

\paragraph*{}
In the near future, most repetitive tasks will inevitably be automated by robots. For instance, in the cleaning industry, fleets of robots will be required to communicate and collaborate effectively to optimise task completion. However, the current generation of robots typically operates in isolation, lacking the ability to be aware of or interact with other robots in their environment, which significantly hinders their overall scalability, efficiency, and coordination capabilities. However, a single complex robot is not viable in every scenario due to its limited reach, and high spatial and monetary investment.

\paragraph*{}
Thus, we turn to a simpler, more scalable solution, which is swarm robotics. Our main goal is to provide a stable foundation for swarm solutions to be applicable for domestic and commercial cleaning. We can achieve this by two major milestones: localisation with object identification, and object grasping through coordinated formations. The first milestone is crucial as it provides a functional component relevant to domestic use cases, enabling object identification, localization, and communication, all of which are key to managing a scalable fleet. Secondly, object grasping through coordinated formations is a crucial follow-up as it expands further upon an abstract concept to pinpoint specific use cases such as tidying items in a room.

\paragraph*{}
The metrics to evaluate our swarm system are scalability, flexibility, architectural tolerance, cost-effectiveness, efficiency, autonomous operations, and performance.

\paragraph*{}
The project involves deploying a swarm of three robots in a room with obstacles and targets, each equipped with RGB-D cameras, LiDAR for GraphSLAM, and ultrasonic sensors, all within a budget of 250,000 THB. Designed with a fixed 90-degree gripper and a two-hour battery life, the robots independently use LiDAR and depth cameras for cost-effective 3D mapping, creating a shared environmental map. Speed is not a critical factor, as the focus is on task completion. Due to the constraints of operating on a level floor, the robots are constructed from heavy materials, which aids in object manipulation. For object detection, the robots utilise RGB-D cameras, and once consensus on the target is reached, roles are dynamically assigned: Alpha and Beta secure the object, while Gamma provides the pushing force to move it to the destination using a sliding method, ensuring efficient 2D movement.

\paragraph*{}
Initially, we considered implementing a C-SLAM (Collaborative Simultaneous Localization and Mapping) approach for our project. However, after further analysis, it became evident that this approach is redundant in our current setup. Each robot in the system is fully capable of independently mapping its environment by using graph SLAM. The added collaboration offered by C-SLAM is not necessary, given that the scope of the project is confined to a single room. The collaborative mapping benefits of C-SLAM would only be justified in scenarios involving multiple rooms, where the robots would need to share information to construct a unified map of a larger, segmented environment.

\paragraph*{}
In the previous semester we were using Xdrive using an ODrive fireware but alas the plan has changed to using the Dynamixel motors instead. This is becuase the ODrive used a SPI interface for the encoder but the motor we had is I2c, in addtion to the retro-fitting new encoder encoder with 3d printed parts, we would have to redesign the base regardless. Hence, changing to Dynamixel motor as they are plug and play out of the box, further reducing the development time, emphasising our focus on software.